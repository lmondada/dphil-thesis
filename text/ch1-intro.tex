% \begin{savequote}[8cm]
% \textlatin{Neque porro quisquam est qui dolorem ipsum quia dolor sit amet, consectetur, adipisci velit...}

% There is no one who loves pain itself, who seeks after it and wants to have it, simply because it is pain...
%   \qauthor{--- Cicero's \textit{de Finibus Bonorum et Malorum}}
% \end{savequote}

\chapter{Introduction}\label{ch:1-intro}

% \minitoc

\section{Motivation}
Quantum circuit compilation is one of the oldest problems in quantum computing.
The problem is to find a sequence of elementary gates that implements a given unitary transformation.
Often stated as a constrained optimisation problem: find a ``quantum circuit'' implementing
the desired computation that minimises a cost function such as the number of gates
of type X, or the \emph{depth} of the computation.

As quantum computers become bigger and computational demands from users increase,
new challenges are emerging in quantum compilation 
\begin{itemize}
  \item Circuits get bigger, making the compilation more computationally expensive.
  \item To scale and improve performance, quantum hardware architecture becomes more complex.
  The result is more special-purpose gate sets, additional execution constraints and more edge cases
  in the compilation problem.
  \item Quantum supercomputing: given the cost and scarcity of quantum computational
  resources, quantum compilation will increasingly be expected to leverage large
  classical compute clusters to lighten as much as possible the quantum resource requirements.
\end{itemize}
The goal of this thesis is to explore new research questions that arise from these developments.
All of these changing demands also come in hand in hand with substantial engineering challenges.

\section{Contributions}
The work we present here contributes towards the scaling up of quantum compilation
in three ways.
\begin{description}
  \item[Quantum Supercomputing.]
    We explore the new challenges for compilation that will arise from larger quantum
    computations, and in particular from the inevitable merging of quantum and classical
    large compute clusters. We explore current tools, their limitations and the
    requirements for future solutions. See \cref{ch:3-quantumsuper}.
  \item[Novel solutions to compilation.]
    Building on a proven \emph{supercompilation} approach, we 
    present two novel results that significantly improve the viability and performance 
    of rewrite rule-based quantum optimisation.
    The first is a fast pattern matching algorithm (\cref{ch:4-pattern}).
    This is semantics-agnostic, meaning that it can be applied not only on quantum circuits
    but also on hybrid quantum-classical computations and even pure-classical Machine Learning
    model training.
    The second is a contribution to circuit optimisation using the ZX-calculus, where we show
    how the pattern matching above can be used in causal-flow preserving ZX optimisations (\cref{ch:5-zx}). 
  \item[Tooling for large scale compilation.]
    We make performance-focused open source infrastructure available that can be
    used to scale compilation to large quantum computations.
    Significant enginnering efforts have gone into the design and implementation of
    this tooling, which we describe in \cref{ch:6-futcompiler}.
\end{description}